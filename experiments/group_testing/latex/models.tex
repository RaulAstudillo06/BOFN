\documentclass[11pt]{article}
%\usepackage{enumitem}   
%\usepackage{graphicx}
\usepackage[margin=1in]{geometry}
\usepackage{comment}
\usepackage{bm}
%\usepackage{algorithm}
\usepackage{algorithmic}
\usepackage{caption}
\usepackage{subcaption}
\usepackage{amsthm}
\usepackage{url}
\usepackage{amsmath}

\newtheorem{theorem}{Theorem}
\newtheorem{definition}{Definition}
\newtheorem{corollary}{Corollary}
\newtheorem{proposition}{Proposition}
\newtheorem{lemma}{Lemma}
\newtheorem{conjecture}{Conjecture}
\newtheorem{assumption}{Assumption}
\newtheorem{fact}{Fact}

\newtheorem*{theorem-no-num}{Theorem}
\newtheorem*{lemma-no-num}{Lemma}

\newcommand{\calUomt}{\calU_{\omega_t}}
\newcommand{\Ugeqt}{U_{\omega_t}^{\geq t}}
\newcommand{\windowsize}{\mathrm{window\_size}}
\newcommand{\minwindow}{\mathrm{min\_window\_size}}
\newcommand{\maxwindow}{\mathrm{max\_window\_size}}
\newcommand{\fluiddiscrepancy}{\mathrm{FluidDiscrepancy}}
\newcommand{\fluiddemand}{\mathrm{FluidDemand}}
\newcommand{\supply}{\mathrm{Supply}}
\newcommand{\demandsupplyratio}{\mathrm{DemandSupplyRatio}}
\newcommand{\demandasymmetry}{\mathrm{DemandAsymmetry}}
\newcommand{\totalutility}{\mathrm{TotalUtility}}
\newcommand{\finalutility}{\mathrm{FinalUtility}}
\newcommand{\utilitygap}{\mathrm{UtilityGap}}
\newcommand{\followfraction}{\mathrm{FollowFraction}}
\newcommand{\finalfollowfraction}{\mathrm{FinalFollowFraction}}
\newcommand{\fluidopt}{\mathrm{FluidOPT}}
\newcommand{\indep}{\perp \!\!\! \perp}
\newcommand{\bfs}{\mathbf{s}}
\newcommand{\bfN}{\mathbf{N}}
\newcommand{\bfM}{\mathbf{M}}
\newcommand{\bfphi}{\bm{\phi}}
\newcommand{\follow}{\mathrm{follow}}
\newcommand{\overcalL}{\overline{\calL}}
\newcommand{\dispatch}{D}
\newcommand{\relocation}{R}
\newcommand{\exit}{E}
\newcommand{\enter}{\mathrm{enter}}
\newcommand{\bbone}{\mathbbm{1}}
\newcommand{\opt}{\mathrm{OPT}}
\newcommand{\bbR}{\mathbb{R}}
\newcommand{\calA}{\mathcal{A}}
\newcommand{\calH}{\mathcal{H}}
\newcommand{\calR}{\mathcal{R}}
\newcommand{\calRgeqt}{\calR_{\omega_t}^{\geq t}}
\newcommand{\calReqt}{\calR_{\omega_t}^t}
\newcommand{\calX}{\mathcal{X}}
\newcommand{\calY}{\mathcal{Y}}
\newcommand{\calQ}{\mathcal{Q}}
\newcommand{\calP}{\mathcal{P}}
\newcommand{\calZ}{\mathcal{Z}}
\newcommand{\calC}{\mathcal{C}}
\newcommand{\calV}{\mathcal{V}}
\newcommand{\calU}{\mathcal{U}}
\newcommand{\calO}{\mathcal{O}}
\newcommand{\calI}{\mathcal{I}}
\newcommand{\calF}{\mathcal{F}}
\newcommand{\calM}{\mathcal{M}}
\newcommand{\calS}{\mathcal{S}}
\newcommand{\calSgeqt}{\calS_{\omega_t}^{\geq t}}
\newcommand{\calSeqt}{\calS_{\omega_t}^t}
\newcommand{\calK}{\mathcal{K}}
\newcommand{\calD}{\mathcal{D}}
\newcommand{\calJ}{\mathcal{J}}
\newcommand{\bbN}{\mathbb{N}}
\newcommand{\bfone}{\mathbf{1}}
\newcommand{\bbE}{\mathbb{E}}
\newcommand{\bmrho}{\bm{\rho}}
\newcommand{\bfm}{\bm{m}}
\newcommand{\bfw}{\bm{w}}
\newcommand{\bfn}{\bm{n}}
\newcommand{\bmk}{\bm{k}}
\newcommand{\bmr}{\bm{r}}
\newcommand{\bfq}{\bm{q}}
\newcommand{\bfx}{\bm{x}}
\newcommand{\bfy}{\bm{y}}
\newcommand{\bfz}{\bm{z}}
\newcommand{\Reg}{\mathrm{Reg}}
\newcommand{\Rel}{\mathrm{Rel}}
\newcommand{\bmI}{\bm{I}}
\newcommand{\bfP}{\bm{P}}
\newcommand{\bfrho}{\bm{\rho}}
\newcommand{\bfr}{\bm{r}}
\newcommand{\bff}{\bm{f}}
\newcommand{\bfh}{\bm{h}}
\newcommand{\bfg}{\bm{g}}
\newcommand{\bfPhi}{\bm{\Phi}}
\newcommand{\bfsigma}{\bm{\sigma}}
\newcommand{\bftheta}{\bm{\theta}}
\newcommand{\bfgamma}{\bm{\gamma}}
\newcommand{\bfalpha}{\bm{\alpha}}
\newcommand{\bfeta}{\bm{\eta}}
\newcommand{\bbP}{\mathbb{P}}
\newcommand{\calL}{\mathcal{L}}
\newcommand{\calT}{\mathcal{T}}
\newcommand{\Rho}{\mathrm{P}}
\newcommand{\calN}{\mathcal{N}}
\newcommand{\calE}{\mathcal{E}}
\newcommand{\calB}{\mathcal{B}}
\newcommand{\calBgeqt}{\mathcal{B}_{\omega_t}^{\geq t}}
\newcommand{\rmtime}{\mathrm{time}}
\newcommand{\overalpha}{\overline{\alpha}}
\newcommand{\underalpha}{\underline{\alpha}}
\newcommand{\overeta}{\overline{\eta}}
\newcommand{\undereta}{\underline{\eta}}

\newcommand{\rev}{\mathrm{reverse}}
\newcommand{\capa}{\mathrm{capacity}}
\newcommand{\weight}{\mathrm{weight}}

%\newcommand{\mccomment}[1]{\textcolor{red}{\textbf{MC:} #1}}
\newcommand{\etcomment}[1]{\textcolor{blue}{\textbf{ET:} #1}}

\newcommand{\currstateset}{\calS_\tau(\omega_\tau)}
\newcommand{\mystateset}[3]{\calS_{#1}^{#2}(#3)}
\newcommand{\stateset}{\calS_\tau^T(\omega_\tau)}

\newcommand{\currrouteset}{\calR_\tau(\omega_\tau)}
\newcommand{\myrouteset}[3]{\calR_{#1}^{#2}(#3)}
\newcommand{\routeset}{\calR_\tau^T(\omega_\tau)}



\newcommand{\frakp}{\mathfrak{p}}
\newcommand{\frakR}{\mathfrak{R}}


\title{Summary of Epidemic Models Used in Group Testing Resource Analysis}
\begin{document}
\maketitle
\section{Introduction}
This note summarizes various epidemic models, paraphrasing heavily from the following
lecture notes \url{https://web.stanford.edu/~jhj1/teachingdocs/Jones-on-R0.pdf}.
The goal is to be precise about the dynamics we are assuming in the group-testing
capacity analysis, and also to potentially study properties of group-testing protocols
through the lens of ODEs.

\section{The Deterministic SIR Epidemic Model}

Consider a closed population of size $N$.  Each individual in the population is
in one of three states: let $S(t)$, $I(t)$, $R(t)$ denote
the number of susceptible, infected, and recovered individuals, respectively, at time $t\geq 0$.

The epidemic dynamics are governed by the following constants:
\begin{itemize}
    \item $\tau=\frac{\mbox{infections}}{\mbox{S-I contact}}$ is the rate at which
        the infection is transmitted per interaction between susceptible and infectious individuals.
    \item $\bar{C} = \frac{\mbox{total number of contacts}}{\mbox{time}}$ is the number of contacts that occur per unit of time.
    \item $\bar{c}=\frac{\bar{C}}{N}$ is the number of contacts per person per time.
    \item $d=\frac{\mbox{infection}}{\mbox{time}}$ is the amount of time that the infection lasts in an infected individual.
\end{itemize}
The above constants are then re-expressed as follows:
$$
\beta = \tau\bar{c}
$$
is the \emph{effective contact rate}, measuring the volume of new infections per unit of time.
$$
\nu = d^{-1}
$$
is the \emph{removal rate}.

We assume a constant population size, constant rates, and a well-mixed population (see above lecture
notes or below discussion).

Let $s(t)=S(t)/N$, $i(t)=I(t)/N$, and $r(t)=R(t)/N$ denote the fraction of individuals in each
state.  The discrete-time SIR model is defined by the following equations:
\begin{align*}
    s(t+1) &= s(t) - \beta i(t) s(t),\\
    i(t+1) &= i(t) + \beta i(t) s(t) - \nu i(t),\\
    r(t+1) &= r(t) + \nu i(t).
\end{align*}
I state the discrete-time dynamics above to make the following interpretation simpler:
The ``well-mixed population'' assumption states that the infection status of both individuals
in each ``contact'' is well approximated by the average infection status in the population.  
Thus, since there are $\bar{C}$ contacts at time $t$, there are $\bar{C}i(t)s(t)$ 
contacts between an infected individual and a susceptible individual at time $t$.
Since $\tau$ is the proportion of new infections per contact between an infected and susceptible
individual, this leads to a volume of 
$$
\tau\bar{C}i(t)s(t) = N \beta i(t) s(t)
$$
new infections at each time $t$, leading to the following discrete-time dynamics:
$$
S(t+1) = S(t) - N\beta i(t)s(t).
$$
Dividing through by the population size $N$, we recover the first equation governing the SIR dynamics:
$$
s(t+1) = s(t) - \beta i(t) s(t).
$$
The interpretation of the remaining equations, governing the discrete-time updates for $i(t)$ and $r(t)$,
should be easy to understand, in light of the assumption that the population is closed, i.e.
$s(t)+i(t)+r(t)=1$ must hold for all $t$.

Moving forward, we will work with the continuous-time analogue of the above dynamics:
\begin{align}
s'(t) &= - \beta i(t) s(t),\\
i'(t) &= \beta i(t) s(t) - \nu i(t),\\
r'(t) &= \nu i(t).
\end{align}
In the absence of an intervention, we see that an infections are growing at a particular time $t$ if
the following criterion is met:
$$
\beta i(t) s(t) - \nu i(t) > 0.
$$
Assuming that the population is almost entirely susceptible, i.e. $s(t)\approx 1$, then this condition
simplifies to
$$
1 < \frac{\beta}{\nu} = \tau\bar{c}d =: \mathcal{R}_0.
$$
Thus, we have derived the standard $\mathcal{R}_0>1$ condition for predicting whether an outbreak
will occur, in a very simple deterministic SIR model. 

\section{Group Testing Intervention in a Deterministic SIR-Q Model}
Consider a ``test-and-quarantine'' intervention that one can run at any time they so choose, possibly subject
to capacity constraints and logistical constraints associated with the test.

To model the dynamics of the quarantine intervention we include
two additional states in the above model: let $Q_I(t)$ denote the number of quarantined, infected
individuals at time $t$, and let $Q_S(t)$ denote the number of quarantined, susceptible individuals
at time $t$.  Let $q_i(t)=Q_I(t)/N$ and $q_s(t)=Q_S(t)/N$ denote their normalized counterparts.

Let $\mathcal{T}$ denote a particular test-and-quarantine protocol, and suppose
each protocol is parameterized by the following metrics:
\begin{itemize}
    \item $QFPR(s,i;\mathcal{T})$ is the \emph{quarantine false-positive rate}.  It has the following interpretation:
        given a proportion of $s=s(t)$ susceptible individuals and $i=i(t)$ infected individuals, what proportion of
        susceptible individuals are forced into quarantine by the test protocol (despite not being infected)?
    \item $QFNR(s,i;\mathcal{T})$ is the \emph{quarantine false-negative rate}.  Given a proportion of
        $s$ susceptible individuals and $i$ infected individuals, what proportion of infected individuals are
        not forced into quarantine by the test protocol (despite that they are infected)?
\end{itemize}


Suppose we are planning to run the test $\mathcal{T}$ at a pre-specified set of times $T\subseteq [0,\infty)$. 
Assume quarantine lasts $d$ days, and all the remaining dynamics are the same as the above model.
The SIR-Q dynamics are then governed by the following equations.  For times $t$ where we do not administer the
test, i.e. $t\notin T$:
\begin{align}
    s'(t) &= -\beta i(t) s(t) + \nu q_s(t),\\
    i'(t) &= \beta i(t) s(t) - \nu i(t) \\
    r'(t) &= \nu i(t) + \nu q_i(t), \\
    q_s'(t) &= -\nu q_s(t) \\
    q_i'(t) &= -\nu q_i(t) .
\end{align}
At times $t$ when we do administer the test, i.e. $t\in T$, we have the following discontinuous dynamics:
\begin{align}
    s(t) &= \left[1-QFPR(s,i;\mathcal{T})\right]s(t),\\
    i(t) &= QFNR(s,k;\mathcal{T}) i(t),\\
    q_s(t) &= \left[1-QFPR(s,i;\mathcal{T})\right]s(t),\\
    q_i(t) &= QFNR(s,k;\mathcal{T})i(t).
\end{align}

\subsection{Optimizing the Test Schedule}
Suppose we want to find a simple schedule that administers the test once every $t_0$ units of time, 
i.e. we want to run the test at each time period $0, t_0, 2t_0, 3t_0$, etc.  What is the largest time
interval $t_0$ we can choose, under which the proportion of infections $i(t)$ nevers grows larger than
its initial value $i(0)$?

Consider the period from time $0$ to $t_0$.  After administering the test at time $0$ the infection volume
is $QFNR(s,k;\mathcal{T}) i(0)$.  What is the longest time $t_0$ we can wait before the infection proportion
surpasses its original volume?  That is, starting with above infection proportion, how many units of time $t_0$
will it take for $i(t_0)=i(0)$?
Given the growth dynamics of $i(t)$ in the absence of testing, this time can be expressed as
$$
QFNR(s,i;\mathcal{T})i(0) + \int_{t=0}^{t_0}\beta i(t)s(t) - \nu i(t)dt = i(0).
$$
Assuming $s(t)\approx 1$ and using the upper bound $i(t)\leq i(t_0)$, we can find a test-frequency $t_0$
which satisfies the tighter bound $i(t_0)\leq i(0)$ by solving the following equation in terms of $t_0$:
$$
QFNR(s,i;\mathcal{T})i(0) + i(0)\int_{t=0}^{t_0}\beta  - \nu dt = i(0).
$$
Rearranging, we have
$$
QFNR(\mathcal{T}) + t_0(\beta - \nu) = 1.
$$
Thus, 
$$
t_0 = \frac{1-QFNR(\mathcal{T})}{\beta - \nu}.
$$
\end{document}
